% !TEX root = ../main.tex

\chapter{内容XXX}

\section{引言}

每章的引言起到承接上一章引启下一章的作用。

\ldots\ldots

\section{普通表格的绘制方法}

表格应具有三线表格式,因此需要调用~booktabs~宏包,其标准格式如表~\ref{table1}~所示。
\begin{table}[htbp]
\caption{符合研究生院绘图规范的表格}
\label{table1}
\vspace{0.5em}\centering\wuhao
\begin{tabular}{ccccc}
\toprule[1.5pt]
$D$(in) & $P_u$(lbs) & $u_u$(in) & $\beta$ & $G_f$(psi.in)\\
\midrule[1pt]
 5 & 269.8 & 0.000674 & 1.79 & 0.04089\\
10 & 421.0 & 0.001035 & 3.59 & 0.04089\\
20 & 640.2 & 0.001565 & 7.18 & 0.04089\\
\bottomrule[1.5pt]
\end{tabular}
\end{table}
全表如用同一单位,则将单位符号移至表头右上角,加圆括号。表中数据应准确无误,书写清楚。数字空缺的格内加横线“-”(占~2~个数字宽度)。表内文字或数字上、下或左、右相同时,采用通栏处理方式,不允许用“〃”、“同上”之类的写法。表内文字说明,起行空一格、转行顶格、句末不加标点。如某个表需要转页接排,在随后的各页上应重复表的编号。编号后加“(续表)”,表题可省略。续表应重复表头。

\section{XXXX分析}

\lipsum[1]

\section{XXXX分析}

\lipsum[2]

\section{本章小结}

总结本章的叙述内容。

\lipsum[3]
